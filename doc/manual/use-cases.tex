\chapter{Use Case: Scientific Simulation} \label{chap:use-cases}

\NOTE{This documentation hasn't been updated to the interface of the upcoming ViennaData 1.1.0 release yet}


The potential of {\ViennaData} for scientific simulations is discussed in this chapter.
More precisely, the use of {\ViennaData} for the solution of partial differential equations by means of discretization methods such as the Finite Element Method or the Finite Volume Method is presented.

It is assumed in the following that the simulation domain, which can be one-, two- or three-dimensional, is decomposed into small elements, say lines, triangles or tetrahedrons. The simulation domains can represent arbitrary objects, for instance buildings, human bones or microelectronic devices, and the physical effects simulated can be displacements, stresses, heat, charge densities, etc.

\section{Handling Material Properties}
Let us consider the classes \lstinline|Line|, \lstinline|Triangle|, and \lstinline|Tetrahedron|, which reflect the respective geometric primitives.
When running a two-dimensional simulation, material properties have to be associated with triangles, thus one might be tempted to add a \lstinline|material| member variable of some type, say, \lstinline|Material| to the \lstinline|Triangle| class.
However, reusing the same class for facets of a \lstinline|Tetrahedron| in three-dimensional simulations would result in an unused
member variable \lstinline|material| in \lstinline|Triangle|, because materials are associated with a \lstinline|Tetrahedron| in three-dimensional simulations.
Similarly, material information should not be a member of a \lstinline|line|.

In order to preserve reusability of the \lstinline|Line|, \lstinline|Triangle| and \lstinline|Tetrahedron| classes, {\ViennaData} can be used as follows for the three-dimensional case:
\begin{lstlisting}
 Tetrahedron tet;
 /* initialize tet */

 //store material on tet using ViennaData:
 viennadata::access<string, Material>("material")(tet) = Material("wood");
\end{lstlisting}
Then, at any later point, the material for \lstinline|tet| can be conveniently retrieved by
\begin{lstlisting}
 Material tet_mat = viennadata::access<string, Material>("material")(tet);
\end{lstlisting}
Other keys can be used as described in Chapter \ref{chap:basic-usage}. Note that one will typically have thousands of cells, not just a single one.

In order to allow a uniform material retrieval independent of the spatial dimension used for the simulation, one can provide a \lstinline|typedef| for the respective cell type:
\begin{lstlisting}
 typedef Tetrahedron     CellType;  //for 3d
 //typedef Triangle      CellType;  //for 2d
 //typedef Line          CellType;  //for 1d

 CellType cell;

 /* Initialize cell here */

 //store data on cell:
 viennadata::access<string, Material>("material")(cell) = Material("wood");

 //retrieve material from cell at some later point:
 Material c_mat = viennadata::access<string, Material>("material")(cell);
\end{lstlisting}
Thus, in this setting {\ViennaData} provides unified access for objects of type \lstinline|Line|, \lstinline|Triangle| and \lstinline|Tetrahedron|.
Moreover, one is not restricted to a particular implementation and can still exchange the implementations for the geometric primitives at some later point without affecting the storage of materials.

\TIP{By using {\ViennaData} one can easily exchange the implementation of the objects the data is associated with.}

\section{Use with Discretization Methods}
The various discretization schemes such as the Finite Volume Method or the Finite Element Method associate unknowns with vertices, lines, facets, and/or cells within a mesh.
A similar reasoning as for the storage of material properties applies to the storage of unknowns (or their identification indices) as well. Moreover, one may not simulate a single effect, but several, which may or may not be coupled. For instance, one may want to simulate mechanical stresses as well as heat flow in a building. Thus, data required for the different simulations may have to be separated. Such a scenario would be almost impossible to handle, if the unknown identification indices were members of the respective classes \lstinline|Line|, \lstinline|Triangle| and \lstinline|Tetrahedron|. However, with {\ViennaData} unknown indices can be stored and retrieved for a point object \lstinline|vertex| as
\begin{lstlisting}
 //store unknown index id_1 for simulation 1 (long sim_1 = 1;):
 viennadata::access<long, UnknownIndex>(sim_1)(vertex) = id_1;
 //store unknown index id_2 for simulation 2 (long sim_2 = 2;):
 viennadata::access<long, UnknownIndex>(sim_2)(vertex) = id_2;

 /* lots of other computations done here */

 //during simulation 1:
 UnknownIndex id_1 = viennadata::access<long, UnknownIndex>(sim_1)(vertex);
 //during simulation 2:
 UnknownIndex id_2 = viennadata::access<long, UnknownIndex>(sim_2)(vertex);
\end{lstlisting}
Thus, not only remains the implementation of geometric primitives light-weight and reusable, the resulting code due to {\ViennaData} is very compact and expressive.
To obtain high execution performance, a dense storage scheme should be employed for accessing the unknown indices, cf.~Section \ref{sec:dense-data}.


Another important aspect for scientific simulation is the handling of boundary conditions. Typically only a subset of elements on the boundary is effected by a certain type of boundary conditions, thus it would be highly inefficient to use a boolean member variable for all elements in the mesh. Here, the default sparse storage scheme of {\ViennaData} turns out to be memory-efficient, when for instance flagging boundary triangles:
\begin{lstlisting}
 // tag tri1, tri42 and tri360 for boundary conditions in simulation 1:
 viennadata::access<long, bool>(sim_1)(tri1)   = true;
 viennadata::access<long, bool>(sim_1)(tri42)  = true;
 viennadata::access<long, bool>(sim_1)(tri360) = true;
 // tag only tri1 and tri360 for boundary conditions in simulation 1:
 viennadata::access<long, bool>(sim_2)(tri1)   = true;
 viennadata::access<long, bool>(sim_2)(tri360) = true;

 /* lots of other computations done here */

 // iterate over triangles using tri_iter in simulation 1
 // and check for boundary conditions:
 if ( viennadata::find<long, bool>(sim_1)(*tri_iter) )
   //boundary triangle for simulation 1
 else
   //no boundary

 // iterate over triangles using tri_iter in simulation 1
 // and check for boundary conditions:
 if ( viennadata::find<long, bool>(sim_2)(*tri_iter) )
   //boundary triangle for simulation 2
 else
   //no boundary
\end{lstlisting}
Note that by using \lstinline|viennadata::access| in the if-clauses, a boolean would be default-constructed and stored on each triangle. By using \lstinline|viennadata::find|, only three and two booleans are stored for the two simulations respectively.

\TIP{{\ViennaData} provides a flexible, unified interface for both sparse and dense storage schemes.}
