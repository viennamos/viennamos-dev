\chapter{Expression Manipulation} \label{chap:manipulation}

The basic description of the types in Chap.~\ref{chap:basics} allows for defining expressions and evaluating them.
However, for most algorithms expressions need to be manipulated in one way or another, which is the topic of this chapter.
Unless otherwise noted, all manipulations considerered in the following can be used for both compiletime and runtime expressions using the same interface.

\TIP{Manipulation functionality resides in folder \lstinline|viennamath/manipulation/|.
 The respective header files are not included automatically with \lstinline|viennamath/expression.hpp| and need to be included as required.}

 \section{Evaluation}
All {\ViennaMath} expressions can be evaluated to a floating point number using the parenthesis operator. 
A vector of values needs to be passed for the evaluation. In the special case that only the first entry of a vector is required for evaluation, it suffices to
pass the value directly.
%Depending on the number of variables in the expression, either a scalar or a vector needs to be passed for evaluation. For conv

Using \lstinline|operator()|, however, is possibly not an option for a generic interface with non-ViennaMath types.
For this reason, \lstinline|viennamath::eval()| provides a generic evaluation interface. 
The first argument is the expression to be evaluated, and the second argument is the tuple with the values to be substituted for the variables.
For example, the expression $x^2$ is defined and evaluated at $x=2$ as follows:
\begin{lstlisting}
 ct_constant<2> c2;
 ct_variable<0> x;
 eval( x*x, 2.0 ); //     runtime evaluation
 eval( x*x,  c2 ); // compiletime evaluation
\end{lstlisting}
Note that compiletime evaluation is only performed when both arguments are fully compiletime compatible.
As soon as one part of the expression cannot be handled at compile time, a fallback to runtime evaluation is carried out.
A hybrid evaluation in such cases is postponed to future releases of {\ViennaMath}.

A vector of values needs to be passed as second argument, if a variable formally refers to any other than the first coordinate in a vector.
Let us consider several use-cases of \lstinline|eval()| consisting of various combinations of compiletime and runtime expressions:
\begin{lstlisting}
 ct_constant<3> c3;
 ct_variable<1> y;
 expr f = x*x + y;     // conversion to runtime expression
 eval( f, 2.0 );       // runtime exception: insufficient values provided
 eval( x*x + y, c2 );  // compilation error: insufficient values provided
 eval( x*x + y,
       make_vector(2.0, 3.0) ); //    runtime evaluation
 eval( x*x + y,
       make_vector(c2, c3) );  // compiletime evaluation
 eval( 2.0, 0.0 );             // possible thanks to overloading
\end{lstlisting}
Since the runtime wrapper \lstinline|expr| hides information from the compiler, an exception is thrown at runtime if insufficient values are provided for evaluation.
For a full compiletime evaluation, insufficient parameters are already detected at an earlier stage.
The helper function \lstinline|make_vector() | generates a suitable vector type both for the runtime and the compiletime case.
Instead of using \lstinline|make_vector()|, a STL vector (\lstinline|std::vector<double>|) or any compatible type can also be passed.
Also note that the last line in the code snippet shows the benefit of using \lstinline|eval()| instead of \lstinline|operator()|:
Scalars can also be 'evaluated' and are thus reinterpreted as constant functions.

 \section{Substitution} \label{sec:substitute}
Formally, the evaluation of an expression can be seen as a substitution of the variables with values.
A generalization is to replace arbitrary expressions with another expression in third expression.
This is accomplished by the function \lstinline|substitute()| defined in \lstinline|viennamath/manipulation/substitute.hpp|:
\begin{lstlisting}
 expr f = (x + y) * (x - y);
 substitute(x, y, f);         // returns (y + y) * (y - y) = 0
 substitute(x, 1, f);         // returns (1 + y) * (1 - y)
 substitute(x + y, x - y, f); // returns (x - y) * (x - y)
\end{lstlisting}
As with \lstinline|eval|, substitutions are carried out at compiletime if all parameters are compiletime expressions.


 \section{Expansion}
It is often desired to expand an expression given as a product of other terms. For example, instead of $2(x+y)$ one may want to have $2x - 2y$.
Such a functionality is provided by the function \lstinline|expand()| defined in \lstinline|viennamath/manipulation/expand.hpp|:
\begin{lstlisting}
 expand( c2 * (x+y) );
\end{lstlisting}
where \lstinline|c2| is a compiletime constant and \lstinline|x|, \lstinline|y| are compiletime variables.
Note that {\ViennaMathversion} does not support the expansion of runtime expressions yet.

\NOTE{ {\ViennaMathversion} supports compiletime expansion only. }



 \section{Simplification}
In the course of manipulating expressions, simple operations such as $x+0$ or $x/1$ may appear.
However, such terms constitute unnecessary overhead for later evaluations, thus it is desirable to have these operations dropped.
Such a simplification of the expression can be achieved with the function \lstinline|simplify()| defined in \lstinline|viennamath/manipulation/simplify.hpp|:
\begin{lstlisting}
 simplify( x + 1.0 * y - 0.0 );  // returns x
 simplify( x * (2.0 + 3.0) + (y * 0) / (x * 1) ); // returns 5x
\end{lstlisting}
\lstinline|simplify()| is available both for runtime and compiletime manipulation.


 \section{Differentiation}
The differentiation of an expression with respect to one or several variables is central to many algorithms, among which the Newton scheme is presumably the
most widely used. Differentiation routines in {\ViennaMath} reside in \lstinline|viennamath/manipulation/diff.hpp| and are used in a canonical way by passing
the expression to be differentiated as first argument and the differentiation variable as second argument:
\begin{lstlisting}
 diff( x + y, x );              // returns 1
 diff( (2.0 - x)*(3.0 + y), y); // returns 2.0 - x
\end{lstlisting}
As usually, compiletime expressions are differentiated at compiletime, while runtime expressions are differentiated at runtime.
An exemplary Newton-solver demonstrating the use of differentiation routines can be found in \lstinline|examples/newton_solve.cpp|.


 \section{Integration}
The integration of an expression over an interval can be accomplished in two ways: The first option is by analytical integration, provided that an
antiderivative of the integrand can be found easily. The second option is by numerical quadrature using a suitable quadrature rule.

 %% Compiletime: Analytic
Analytical integration is in {\ViennaMathversion} available for compiletime types and polynomials as integrands only.
The file \lstinline|viennamath/manipulation/integrate.hpp| provides the function \lstinline|integrate()|, which takes the integration interval
as first argument, the integrand as second argument and the integration variable as third argument. For example, the integral
\begin{align}
 \int_0^1 x^2 \: \mathrm{d} x
\end{align}
is computed analytically at compile time using the lines
\begin{lstlisting}
 integrate( make_interval(c0, c1),
            x * x,
            x);              //returns the expression 1/3
\end{lstlisting}
where \lstinline|c0| and \lstinline|c1| denote the compiletime constants $0$ and $1$ and are passed to the helper function \lstinline|make_interval()|
to generate a suitable compiletime interval with the provided lower and upper bound. Note that analytic integration can also be nested and use polynomial lower
and upper bounds. For example, integration of $x^2$ over the unit triangle with vertices $(0,0)$, $(1,0)$ and $(0,1)$ is achieved via
\begin{lstlisting}
 integrate( make_interval(c0, c1),
            integrate( make_interval(c0, c1 - x),
                       x * x,
                       y),
            x);              //returns 1/12
\end{lstlisting}


\NOTE{ Analytic integration in {\ViennaMathversion} is available for polynomial integrands at compiletime only. }



 %% Runtime: Quadrature
\begin{table}
\centering
\begin{tabular}{|l|l|l|l|}
\hline
Name & {\ViennaMath} Type   & Shortcut & Accuracy \\
\hline
$1$-point Gauss   & \lstinline|rt_gauss_quad_1| & \lstinline|gauss_quad_1| & $1$ \\
\hline
\end{tabular}
\caption{Overview of numerical quadrature rules in {\ViennaMath}. \label{tab:quadrature}}
\end{table}

In order to compute an integral numerically using a quadrature rule, the respective rule from Tab.~\ref{tab:quadrature} needs to be instantiated first.
For a $1$-point Gauss rule, this is accomplished by
\begin{lstlisting}
 rt_numerical_quadrature<InterfaceType> 
    integrator(new rt_gauss_quad_1<InterfaceType>());
\end{lstlisting}
for a suitable runtime interface type. If the default interface type is to be used, the shortcut types can be used for convenience:
\begin{lstlisting}
 numerical_quadrature integrator(new gauss_quad_1());
\end{lstlisting}
To carry out the numerical quadrature, two options exist: The first is to pass the integration interval, the integrand, and
the integration variable as separate arguments to \lstinline|operator()| of the \lstinline|integrator| object:
\begin{lstlisting}
 integrator( make_interval(0.0, 1.0),
             x * x,
             x);             // returns the value 0.25
\end{lstlisting}
The second option for numerical quadrature is to encode the integral directly in the expression:
\begin{lstlisting}
  expr my_integral = integral( make_interval(0, 1), x * x, x );
\end{lstlisting}
which encodes $\int_0^1 x^2 \: \mathrm{d} x$ directly in an expression. For numerical quadrature, only the encoded form needs to be passed to the quadrature
rule then
\begin{lstlisting}
 integrator(my_integral);    // returns the value 0.25 (1-point Gauss)
\end{lstlisting}



 \section{Extract Coefficient}
Given a polynomial $p(x,y) = 1 + x + y + 2xy$ it can be of interest to extract individual parts of the polynomial.
For example, one wishes to extract the coefficient of $x$, which is $(1+2y)$. In such a case, 
the function \lstinline|coefficient()| defined in \lstinline|viennamath/manipulation/coefficient.hpp| can be used.
The first parameter is the variable or expression for which the coefficient should be returned, and the second argument is the expression from which the
coefficient is to be extracted:
\begin{lstlisting}
 coefficient(x, c1 + x + y + c2*x*y);  //returns 1+2y
\end{lstlisting}
Note that higher-order terms in the variable are also returned. For example, the coeffient of $x$ in $x+x^2$ is obtained as $(1+x)$.

\NOTE{ \lstinline|coefficient()| is in {\ViennaMathversion} available for compiletime types only. }


 \section{Drop Dependent Terms}
In order to drop all terms in an expression which depend on a certain expression type (not necessarily a variable), the convenience function
\lstinline|drop_dependent_terms()| from \lstinline|viennamath/manipulation/drop_dependent_terms.hpp| can be used. As the name suggests, all terms with a
dependency on the expression passed as first parameter are dropped in the expression passed as second variable.
For example, all terms depending on $x$ in $1+x+y+2xy$ are dropped using the line
\begin{lstlisting}
 drop_dependent_terms(x, c1 + x + y + c2*x*y);  //returns 1+y
\end{lstlisting}
in order to obtain $1+y$.



\NOTE{ \lstinline|drop_dependent_terms()| is in {\ViennaMathversion} available for compiletime types only. }


